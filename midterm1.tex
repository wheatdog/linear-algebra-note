%
% Homework Details
% - Title
% - Due date
% - University
% - Class
% - Class Alias
% - Class Section
% - Instructor
% - Author
% - AuthorID
%

\documentclass[9pt, twocolumn]{extarticle}

%
% Packages
%

\usepackage[a4paper,margin=0.5in,landscape]{geometry}
\usepackage{amssymb}
\usepackage{amsmath}
\usepackage[inline]{enumitem}

%
% Chinese
%

\usepackage{xeCJK}
\setCJKmainfont{Noto Sans CJK TC Regular}

\XeTeXlinebreaklocale "zh"
\XeTeXlinebreakskip = 0pt plus 1pt

%
% Basic Document Settings
%

%\topmargin=-1.00in
%\evensidemargin=0in
%\oddsidemargin=0in
%\textwidth=6.5in
%\textheight=10.0in
%\headsep=0.25in
%
%\linespread{1.0}

\begin{document}

\begin{description}
    \item[5.9  特徵基底判別定理] Let $V$ be a vector space with $dim(V) = n$. If $T \in \mathcal{L}(V)$ and $\lambda_1 \dots, \lambda_k$ are the distinct eigenvalues of $T$, then $T$ is diagonalizable if and only if $dim(E_T(\lambda_1)) + \cdots + dim(E_T(\lambda_k)) = n$.
    \item[5.2  特徵值 $\Leftrightarrow$ 特徵根] If $A \in F^{n \times n}$, then $\lambda$ is an eigenvalue of $A$ if and only if $f_A(\lambda) \equiv det(A - \lambda I_n) = 0_F$.
    \item[5.4  特空定理] If $T$ is linear operator on vector space $V$ and $\lambda$ is an eigenvalue of $T$, then $T(x) = \lambda x \Leftrightarrow x \in E_T(\lambda)$.
    \item[5.7  特徵值重數定理] Let $V$ be a finite-dimensional vector space. If $\lambda$ is an eigenvalue of $T \in \mathcal{L}(V)$ with multiplicity $m$, then $1 \leq dim(E_T(\lambda)) \leq m$.
    \item[5.5  跨特徵空間不冗定理] Let $T$ be a linear operator on an $n$-dimensional vector space $V$. Let $\lambda_1, \dots, \lambda_k$ be distinct eigenvalues of $T$. If $\varnothing \neq S_i \subseteq E_T(\lambda_i)$ is linearly independent for each $i = 1, \dots, k$, then $S_1, \dots, S_k$ are pairwise disjoint and $S_1 \cup \dots \cup S_k$ is linearly independent.
    \item[5.11 特徵空間直和定理] If $T \in \mathcal{L}(V)$ for a finite-dimensional vector space $V$, then $V$ is the direct sum of the eigenspaces of $T$ is and only if $V$ has an eigenbasis for $T$.
    \item[5.23 Cayley-Hamilton Theorem] If $T \in \mathcal{L}(V)$ for a finite-dimensional vector space $V$ over $F$, $f_T(T) = T_0$.
    \item[5.21 縮水觀察] Let $T \in \mathcal{L}(V)$ for vector space $V$ with $dim(V) < \infty$. If $U$ is a $T$-invariant subspace of $V$, then $f_{T_U}(t) | f_T(t)$.
    \item[5.22 循環定理] Let $T \in \mathcal{L}(V)$ with $dim(V) < \infty$. Let $U = C_T(x) \equiv span\left(\cup_{i \leq 0} T^i(x) \right)$ with $x \in V \backslash \{ 0_V \}$. Let $k = dim(U)$.
        \begin{enumerate*}[label=(\alph*),itemjoin={;\quad}]
            \item The ordered set $\beta = \langle x, T(x), \dots, T^{k-1}(x) \rangle$ is a basis of $U$
            \item If $\sum_{i=0}^k a_i T^i(x) = 0_V$ with $a_k = 1_F$, then $f_{T_U}(t) = (-1)^k \sum_{i=0}^k a_i t^i$
        \end{enumerate*}
    \item[Definition 內積函數] 首項線性、共軛對稱、正定
    \item[6.1 內積基本性質]
    \item[6.3 正交定理] Let $x$ be a vector in inner-product space $V$. Let $S$ be a nonempty orthogonal set of nonzero vectors in $V$. Let $R$ be a finite subset of $S$. If $x = \sum_{y \in R} a_y y$ holds for scalars $a_y$, then $a_y = \frac{\langle x | y \rangle}{\langle y | y \rangle}$ holds for each $y \in R$
    \item[正交無零則不冗]
    \item[6.4 正交演算法] For any linearly independent subset $\alpha$ of inner-product space $V$ with $|\alpha| = n$, the set $\beta$ recursively defined below is orthogonal basis of $span(\alpha)$: $\beta_1 = \alpha_1$ and $\beta_j = \alpha_j - \sum_{i=1}^{j-1} \frac{\langle \alpha_j | \beta_i \rangle}{\langle \beta_i | \beta_i \rangle} \dot \beta_i$ for each $j = 2, \dots, n$.
    \item[6.2 長度的基本性質]
    \item[Definition 正交補集] For any nonempty subset $S$ of inner-product space $W$, the orthogonal complement of $S$ is $S^\perp \equiv \{ x \in W: \langle x | y \rangle = 0_F \text{ holds for all } y \in S \}$.
    \item[正補定理] If $V$ is a subspace of an inner-product space $W$ with $dim(W) < \infty$, then $V \bigoplus V^\perp = W$.
    \item[6.25 特徵值譜定理] Let $T \in \mathcal{L}(V)$ for inner-product space $V$ with $dim(V) < \infty$ that has an orthonormal eigenbasis for $T$. Let $\lambda_1, \dots, \lambda_k$ be the distinct eigenvalues of $T$.
        \begin{enumerate}
            \item For $i \in \{ 1, \dots, k \}$, $E_T(\lambda_i)^\perp = E_T(\lambda_1) \bigoplus \dots \bigoplus E_T(\lambda_{i-1}) \bigoplus E_T(\lambda_{i+1}) \bigoplus \dots \bigoplus E_T(\lambda_k)$
            \item If each $T_i$ with $1 \leq i \leq k$ is the orthogonal projection of $V$ on $E_T(\lambda_i)$, then
                \begin{enumerate}
                    \item For indices $1 \leq i$, $j \leq k$, if $i = j$, then $T_i T_j = T_i$; if $i \neq j$, $T_i T_j = T_0$
                    \item (Resolution of $I_V$) $T_1 + T_2 + \dots + T_k = I_V$
                    \item (Spectral decomposition of $T$) $\lambda_1 T_1 + \lambda_2 T_2 + \dots + \lambda_k T_k = T$
                \end{enumerate}
        \end{enumerate}
    \item[正交投影] A projection $T$ of inner-product space $W$ is orthogonal if $T(W)^\perp = N(T)$; $N(T)^\perp = T(W)$.
    \item[6.8 泛函定理] For any functional $f$ on an inner-product space $V$ with $dim(V) < \infty$, there is a unique vector $y \in V$ such that $f(x) = \langle x | y \rangle$ holds for all $x \in V$.
    \item[6.9 翻牆定理] For any linear operator $T$ on any inner-product space $V$ with $dim(V) < \infty$, there is a unique operator $T^*$ on $V$ such that $\langle T(x) | y \rangle = \langle x | T^* (y) \rangle$ holds for all vectors $x, y \in V$. Moreover, this unique $T^*$ is linear.
    \item[6.10 翻牆推論] Let $V$ be a finite-dimensional inner-product space. The following statements hold for any linear operator $T$ on $V$.
        \begin{enumerate*}[itemjoin={;\quad}]
            \item For any $x, y \in V$, $\langle x | T(y) \rangle = \langle T^*(x) | y \rangle$
            \item For any orthonormal basis $\beta$ of $V$, $[ T^* ]^{\beta}_{\beta} = ([ T ]^{\beta}_{\beta})^*$
        \end{enumerate*}
    \item[6.11 伴隨線運基本性質] Let $V$ be an inner-product space over $F$ with $dim(V) < \infty$. The following equations hold for any $T_1, T_2, T \in \mathcal{L}(V)$ and any $a \in F$:
        \begin{enumerate*}[itemjoin={;\quad}]
            \item $(a T_1 + T_2)^* = \bar{a} T_1^* + T_2^*$
            \item $(T_1T_2)^* = T_2^* T_1^*$
            \item $(T^*)^* = T$
            \item $I_V^* = I_V$
        \end{enumerate*}
    \item[Observation 伴隨線運的特徵值] If $\lambda$ is an eigenvalue of $T$, then $\bar{\lambda}$ is an eigenvalue of $T^*$
    \item[Definition 常態線運、常態方陣] $T T^* = T^* T$; $A A^* = A^* A$
    \item[常態小觀察] $T$ is normal if and only if $[ T ]_{\beta}^{\beta}$ is normal
    \item[6.16 值譜證明起手式] If $V$ has an orthonormal eigenbasis for $T$, then $T$ is normal
    \item[6.15 常態線運基本性質]
        \begin{enumerate*}[itemjoin={;\quad}]
            \item For each $x \in V$, $\| T(x) \| = \| T^*(x) \|$
            \item For each $a \in F$, $T + a I_V$ is normal
            \item If $(\lambda, x)$ is an eigenpair of $T$, then $(\bar{\lambda}, x)$ is an eigenpair of $T^*$
            \item If $x$ and $y$ are in distinct eigenspaces of $T$, then $\langle x | y \rangle = 0_F$
        \end{enumerate*}
    \item[Corollary 4 特徵值譜推論] If $V$ has an orthonormal eigenbasis for $T$, then the othogonal projection $T_i$ of $V$ on $E_T(\lambda_i)$ is a polynomial in $T$, where $\lambda_i$ is the $i$-th distinct eigenvalue of $T$.
    \item[6.14 舒爾定理] If $f_T(t)$ splits, then there is an orthonormal basis $\beta$ of $V$ such that $[T]_{\beta}^{\beta}$ is upper triangular
        
\end{description}
\end{document}
