%
% Homework Details
% - Title
% - Due date
% - University
% - Class
% - Class Alias
% - Class Section
% - Instructor
% - Author
% - AuthorID
%

\documentclass[8pt, twocolumn]{extarticle}

%
% Packages
%

\usepackage[a4paper,margin=0.5in,landscape]{geometry}
\usepackage{amssymb}
\usepackage{amsmath}
\usepackage[inline]{enumitem}

%
% Chinese
%

\usepackage{xeCJK}
\setCJKmainfont{Noto Sans CJK TC Regular}

\XeTeXlinebreaklocale "zh"
\XeTeXlinebreakskip = 0pt plus 1pt

%
% Basic Document Settings
%

%\topmargin=-1.00in
%\evensidemargin=0in
%\oddsidemargin=0in
%\textwidth=6.5in
%\textheight=10.0in
%\headsep=0.25in
%
%\linespread{1.0}

\newcommand{\vsdim}{\ensuremath{\text{dim}}}
\newcommand{\rank}{\ensuremath{\text{rank}}}
\newcommand{\realnum}{\mathbb{R}}
\newcommand{\complexnum}{\mathbb{C}}
\newcommand{\ltrans}{\mathbb{L}}
\newcommand{\polynom}{\mathbb{P}}

\begin{document}

\begin{description}
        % 5th slide
    \item[6.16 常態定理] Let $T\in \ltrans(V)$ for inner-product space $V$ over $\complexnum$ with $\vsdim(V) < \infty$. $V$ has an orthonormal eigenbasis for $T$ iff $T$ is normal, i.e. $TT^*=T^*T$
    \item[Def 自伴線運、方陣] T is self-adjoint if $T^* = T$. Square matrix $A$ is Hermitian if $A^*=A$
    \item[Obs 自伴小觀察] If $\beta$ is an orthonormal basis of a inner-product space $V$ with $\vsdim(V) < \infty$, then $T$ is self-adjoint iff $[T]^\beta_\beta$ is self-adjoint.
    \item[Obs 自伴線運基本性質] If $T$ is a self-adjoint linear operator on an inner-product space $V$ over $F \in \{\realnum, \complexnum\}$ with $\vsdim(V)<\infty$, then
	    \begin{enumerate*}
		    \item each eigenvalue of $T$ is real (even if $F = \complexnum$, and)
		    \item the characteristic polynomial $f_T (t)$ of $T$ splits (even if $F = \realnum$)
	\end{enumerate*}
    \item[6.24 投影:正交$\Leftrightarrow$自伴] If $T$ is a projection of $W$, then $T$ is an orthogonal projection of $W$ iff $T$ is self-adjoint
    \item[Cor 自伴線運推論] Let $T \in \ltrans(V)$ for inner-product space $W$ over $F = \complexnum$ wth $\vsdim(V) < \infty$. If $T$ is normal, $T$ is self-adjoint iff every eigenvalue of $T$ is real.
    \item[6.17 自伴定理] Let $T \in \ltrans(V)$ for $V$ over $\realnum$ with $\vsdim(V)<\infty$. $V$ has an orthonormal eigenbasis for $T$ iff T is self-adjoint
    \item[Cor 常態線運推論] If $T\in \ltrans(V)$ for $W$ over $\complexnum$ with $\vsdim(W)< \infty$, then $T$ is normal iff $T^* = g(T)$ for some polynomial $g \in \polynom(\complexnum)$


        % 6th slide
    \item[Def 么正、正交方陣] Let $Q \in F^{n\times n}$ with $F \in \{\complexnum, \realnum\}$,
	    \begin{itemize*}
		    \item Q is unitary if $Q^*Q = I_n$ (i.e. $Q^*=Q^{-1}$)
		    \item Q is orthogonal if $Q^t Q = I_n$ (i.e. $Q^t = Q^{-1}$)
	    \end{itemize*}
    \item[Def 么正、正交等價] Let $A, B \in F^{n \times n}$ with $F \in \{\complexnum, \realnum\}$,
	    \begin{itemize*}
		    \item $A$ is unitarily equivalent to $B$ if there is a unitary matrix $Q$ with $A = Q^*BQ$
		    \item $A$ is orthogonally equivalent to $B$ if there is an orthogonal matrix $Q$ with $A = Q^tBQ$
	    \end{itemize*}
    \item[6.20] If $A \in \realnum^{n\times n}$, then $A$ is self-adjoint (i.e. symmetric) iff $A$ is orthogonally (i.e. unitarily) equivalent to a diagonal matrix in $\realnum^{n\times n}$
    \item[6.19] If $A \in \complexnum^{n\times n}$, then $A$ is normal iff $A$ is unitarily equivalent to a diagonal matrix in $\complexnum^{n\times n}$
   
    \item[6.21 方陣舒爾] If $f_A(t)$ splits for $A \in F^{n\times n}$, then $A$ is unitarily equivalent to an upper-triangular matrix in $F^{n\times n}$.
        
    \item[Def 么正、正交線運] Let $T \in \ltrans(V)$, $V$ is an inner-product space over $F\in \{\complexnum, \realnum\}$. 
        \begin{itemize*}
            \item $T$ is unitary if $T^*T = I_V$
            \item $T$ is orthogonal if $T$ is unitary and $F = \realnum$
        \end{itemize*}
    \item[6.18] If $T \in \ltrans(V)$ for $V$ over $F\in \{\complexnum, \realnum\}$, with $\vsdim(V)<\infty$, then the following are equivalent:
        \begin{itemize*}
            \item $T^*T=I_V$
            \item $\langle T(x) | T(y)\rangle = \langle x | y \rangle$ holds for all vectors $x, y \in V$
            \item For any orthogonal eigenbasis $\beta$ of $V$, $T(\beta)$ is an orthonormal basis of $V$
            \item There is a $\beta \subseteq V$ s.t. $\beta$ and $T(\beta)$ are both orthonormal bases of $V$
            \item $\| T(x)\| = \|x \|$ holds for all vectors $x \in V$
    
        \end{itemize*}

    \item[Cor 么正、正交自伴定理] If $T \in \ltrans(V)$ for $V$ over $\complexnum$ ($\realnum$) with $\vsdim(V) < \infty$, then $T$ is unitary (orthogonal and self-adjoint) iff 
        \begin{itemize*}
                \item $V$ has an orthonormal eigenbasis for $T$
                \item each eigenvalue of $T$ has absolute value 1
        \end{itemize*}
    \item[6.13 最短解] If $E: Ax = b$ with $A \in F^{m\times n}$ and $b \in F^m$ is a system of linear equations with $S(E) \neq \emptyset$, then there is exactly one vector $x$ in $S(E) \cap L_{A^*}(F^m)$ w.r.t. the standard inner product. Moreover, the vector $x$ is the unique vector in $S(E)$ with minimum $||x||$
    \item[6.12 最佳近似解] Let $A \in F^{m\times n}$ and $b \in F^{m}$.
        \begin{itemize*}
                \item For any inner-product function of $F^m$, $\exists x \in F^n$ that minimizes $\|Ax-b\|$
                \item   If $\rank(A) = n$, then $x = (A^*A)^{-1}A^*b$ is the unique minimizer of $\|Ax-b\|$ w.r.t. the standard inner product of $F^m$
        \end{itemize*}
    \item[Obs 標準內積觀察] Let $A \in F^{m\times n}$. For any $x \in F^n$ and $y \in F^m$, we have $\langle Ax | y \rangle_m = \langle x | A^* y \rangle_n$
    \item[Obs 矩陣位階觀察] For any $A \in F^{m\times n}$, $\rank(A^*A)=\rank(A)$
    \item[Obs 伴隨矩陣觀察] For any $A \in F^{m\times n}$, if $\rank(A) = n$, then $A^*A\in F^{n\times n}$ is invertible.
        
        % 7th slide 
    \item[Def 正定] A self-adjoint $T \in \ltrans(V)$ for $V$ over $F$ is positive definite (semidefinite) if $\langle T(x)|x\rangle \in \realnum^+$ ($\langle T(x)|x\rangle \in \realnum^+ \cup \{0_\realnum\}$) holds for all $x \in V \backslash \{0_V\}$ ($x \in V$)
    \item[Obs (半)正定線運等價條件] If $T \in \ltrans(V)$ with $\vsdim(V)<\infty$ is self-adjoint, then $T$ is positive (semi)definite iff all eigenvalues are positive (non-negative) real numbers
    \item[Obs 半正定方陣等價條件] For $F \in \{\complexnum, \realnum\}$, $A \in F^{n\times n}$ is positive semidefinite iff $A = B^*B$ holds for some $B \in F^{n\times n}$
    \item[Obs 半正定平方性質] If $T_1, T_2 \in \ltrans(V)$ ($A, B \in F^{n\times n}$) with $T_1^2=T_2^2$ ($A^2 = B^2$), then $T_1 = T_2$ ($A = B$)
    \item[Def 伴隨線轉] $T \in \ltrans(V, W)$. $T^*: W \rightarrow V$ is an \textit{adjoint} of $T$ if  $\langle T(x)|y\rangle_W = \langle x | T^*(y)\rangle_V$ holds $\forall x \in V, y \in W$
    \item[Obs 伴隨線轉觀察] If $T \in \ltrans(V, W)$ with $\vsdim(V)<\infty, \vsdim(W)<\infty$ then, 
        \begin{itemize*}
                \item Both $T^*T$, $TT^*$ are positive semidefinite
                \item $N(T^*T)=N(T)$
                \item $\rank(T^*T) = \rank(TT^*)=\rank(T)=\rank(T^*)$
        \end{itemize*}
    \item[6.26 奇異值定理] Let $T \in \ltrans(V, W)$ for $V$ and $W$ over $F \in \{\complexnum, \realnum\}$ with $\rank(T)=r$, $\vsdim(V) = n$, and $\vsdim(W) = m$.
        \begin{enumerate*}[label=({\alph*})]
            \item There exist a set $\sigma = \langle \sigma_1, \cdots, \sigma_r \rangle$ of positive real numbers and orthonormal bases $\beta  = \langle \beta_1, \cdots , \beta_n \rangle$ of $V$ and $\gamma = \langle \gamma_1, \cdots , \gamma_m \rangle$ of $W$ s.t. with $\sigma_k = 0_F$ for $r+1 \leq k \leq \max(n, m)$ and $\gamma_k = 0_W$ for $m+1 \leq k \leq n$ 
                \begin{enumerate*}[label={\arabic*}.]
                    \item $T(\beta_j) = \sigma_j \gamma_j$ holds for each $j = 1, \cdots, n$. 
                \end{enumerate*}
            \item Moreover, the following hold for any such $\sigma,\beta,\gamma$:
                \begin{enumerate*}[label={\arabic*}.]
                        \setcounter{enumii}{1}
                        \item $T^*(\gamma_i) = \sigma_i \beta_i$ holds for $i = 1,\cdots, m$ with $\beta_k = 0_V$ for $n+1 \leq k \leq m$
                        \item Each $(\sigma^2_j, \beta_j)$ with $1 \leq j \leq n$ is an eigenpair of $T^*T$
                        \item Each $(\sigma^2_i, \gamma_i)$ with $1 \leq i \leq m$ is an eigenpair of $TT^*$
                \end{enumerate*}
        \end{enumerate*}
    \item[6.27 SVD] For any matrix $A = F^{m\times n}$ with rank $r$, $A = QSR*$ holds for: 
        \begin{itemize*}
            \item unitary matrices $Q \in F^{m\times m}$ and $R \in F^{n\times n}$
            \item a diagonal matrix $S \in F^{m\times n}$ whose $l = \min (m, n)$ diagonal elements $\sigma_i = S_{i, i}$ with $i \in \{1,\cdots ,l\}$ satisfy 
                \begin{enumerate*}
                    \item $\sigma_1, \cdots , \sigma_r \in \realnum^+$
                    \item $\sigma_{r+1} = \cdots = \sigma_l = 0_F$
                \end{enumerate*}
        \end{itemize*}
        
        
        % 8th slide
    \item[Def 偽反線轉] $T^\dagger \in \ltrans(W, V)$ defined by $T^\dagger = (T')^{-1}T''$
    \item[偽反線轉定理] Let $T \in \ltrans(V,W)$ for $V$ and $W$ over $F \in \{\complexnum, \realnum\}$ with $\rank(T)=r, \vsdim(V)=n,\vsdim(W)=m$. If $\sigma, \beta, \gamma$ are as ensured by \textbf{6.26}, then $T^\dagger$ is the unique function in $\ltrans(W, V)$ satisfying $T^\dagger(\gamma_j)=\left\{\begin{array}{cl} \frac{1}{\sigma_j}\beta_j & \text{if} 1 \leq j \leq r\\ 0_V & \text{otherwise}\end{array}\right.$
	\item[偽反矩陣觀察] $S^\dagger$ 對角線上非零的值是原本的倒數,另外 $S^\dagger$ 的行數和列數會互換。
    \item[偽反線運等價條件] If $T_1 \in \ltrans(V, W)$ and $T_2 \in \ltrans(W, V)$ for $V, W$ over $F \in \{\complexnum, \realnum\}$ and $\vsdim < \infty$, then $T_2 = T_1^\dagger$ iff \textbf{all} following hold:
        \begin{enumerate*}[label=({\alph*})]
            \item $T_1T_2T_1=T_1$
            \item $T_2T_1T_2=T_2$
            \item $T_1T_2$ and $T_2T_1$ are self-adjoint
        \end{enumerate*}
    \item[偽反矩陣定理] If $A = QSR^*$ is a SVD of matrix $A \in F^{m\times n}$ with $F \in \{\complexnum, \realnum\}$, then $A^\dagger = RS^\dagger Q^*$ is a SVD of matrix $A^\dagger \in F^{n\times m}$
    \item[6.30] For any system of linear equations $E: Ax = b$ with $A \in F^{m\times n}$, $y = A^\dagger b$ is the unique vector in $F^n$ with 
        \begin{itemize*}
            \item $\|Ay-b\|\leq \|Ax - b\|$ for any $x \in F^n$ and 
            \item $\|y\| < \|x\|$ any $x \in F^n \backslash \{y\}$ with $\|Ay-b\| = \|Ax-b\|$
        \end{itemize*}
    \item[6.28 極分解定理] For any square matrix $A \in F^{n\times n}$ with $F \in \{\complexnum, \realnum\}$, there exist a unitary matrix $Q \in F^{n\times n}$ and a positive semidefinite matrix $P \in F^{n\times n}$ s.t. $A = QP$. And, if $A$ is invertible, then the decomposition is unique ($Q = Q_0R^*, P = RSR^*$)

\end{description}
\end{document}
